
\chapter*{Foreword}\addcontentsline{toc}{chapter}{Forword}

If I were to ask you what you think an electrical engineer is or does, you might answer “An electrical engineer is someone who solves problems that involve electricity.” But, if a friend replaces a light bulb, does that make my friend an electrical engineer? If you intuitively answer no, even if you can’t quite express why, that’s OK. One of my goals for you in this course is to help you to better understand what an electrical engineer is and does, and as a result be able to explain why my friend is not an electrical engineer (with which I agree, by the way).

One of the primary differences between engineers and people who solve problems \textit{ad hoc} (which by the way is everyone, in some sense) is that engineers tend to use a systematic, strategic approach to problem-solving, which includes the ability to test the problem solution to ensure that 1) you solve the correct problem, and 2) you solved the problem correctly. Another of my goals for you is to develop such a strategic, systematic approach to solve problems whose solutions involve processing information of an electrical nature.

But, a good solution to a problem is only as good as your ability to communicate it clearly, concisely, and effectively to others. Another goal of mine for you is to develop such communication skills, both in writing and in speaking, so that others will read and hear what you intend to communicate.

So, why bother with all of this?  After all, it sounds like a lot of really difficult work – and it is. But, my overarching goal for you in this course is for you to have the knowledge, skills, and the habit of careful, rigorous attention to detail that make you more attractive to potential employers at the end of the semester than at the beginning.  After all, light bulbs aren’t free!

\hfill \textemdash Todd DeLong