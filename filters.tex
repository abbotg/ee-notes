\chapter{Filters}

\section{Second-Order Low Pass}
Generalized equation:
\[H(s)=\frac{K}{1+\left(\frac{s}{\omega_0}\right)\frac{1}{Q} + \left(\frac{s}{\omega_0}\right)^2} = \frac{\omega_0^2}{s^2 + \frac{\omega_0}{Q}s + \omega_0^2}\]

\section{Quality Factor \& Damping}
Tells us the quality of our energy storage components and how close they are to their ideal versions.
\[Q=\frac{\text{Energy stored}}{\text{Energy lost per cycle}}\]
Energy storage equations for a capacitor and an inductor:
\[U_C=\frac{1}{2}CV^2 \quad U_L=\frac{1}{2}LI^2\]
\subsection{Derivation of Q factor for an RLC circuit}
Energy stored in the inductor:
\[U=\frac{1}{2}LI^2\]
Energy dissipated in the resistor:\footnote{I dont know where this equation comes from}
\[U=\frac{1}{2}I^2 R/\omega_0\]
So the quality factor is:
\[Q=\frac{\frac{1}{2}LI^2}{\frac{1}{2}I^2 R/\omega_0}=\frac{\omega_0 L}{R}\]
Since the resonant frequency is:
\[\omega_0=\frac{1}{\sqrt{LC}}\]
The Q factor for an RLC circuit is
\[Q=\frac{1}{R}\sqrt{\frac{L}{C}}\]

\subsection{Damping}
Damping is a time domain phenomenon, and refers to the step response.
\begin{align*}
Q<0.5:&\text{ Overdamped}\\
Q>0.5:&\text{ Underdamped}\\
Q=0.5:&\text{ Critically damped}\\
\end{align*}




\section{Butterworth Characteristic}
For a second-order low/high pass filter, this is the Q factor that makes the frequency response maximally flat. The value is:
\[Q=\frac{1}{\sqrt{2}}=\frac{\sqrt{2}}{2}=0.707\]

\section{Sallen-Key, Second-Order}
Equations:\footnote{$Q$ and $\omega_0$ are found by comparing the transfer function to standard form}
\begin{gather*}
H(s)=\frac{V_{out}}{V_{in}}=\frac{K}{s^2 R_1 R_2 C_1 C_2 + s\left[R_1 C_2 + R_2 C_2 + R_1 C_1\left(1-K\right)\right]+1}\\
\omega_0=\frac{1}{\sqrt{R_1 R_2 C_1 C_2}} \qquad Q = \frac{\sqrt{R_1 R_2 C_1 C_2}}{R_1 C_2 + R_2 C_2 + R_1 C_1\left(1-K\right)}
\end{gather*}
Designing a Sallen-Key filter:
\begin{enumerate}
	\item Determine the gain of the system, $K$. If unity-gain, $K=1$. Values of $R_A$ and $R_B$ can be determined here.
	\item Usually given a corner frequency and a Q factor, e.g. $\frac{\sqrt{2}}{2}=0.707$. Use the ratio method to find values of $m$ and $n$ that equal this Q factor:
	\[Q=\frac{\sqrt{mn}}{m+1}\]
	\item Choose values of $R$ and $C$ to match the desired corner frequency using these new values $m$ and $n$:
	\[\omega_0=\frac{1}{\sqrt{mn}RC}\]
	\item Then use the ratios to find the values of $R_1$ and $C_1$:
	\begin{align*}
	R_2 &= R & C_2 &= C\\
	R_1 &= mR & C_1 &= nC \\
	\end{align*}
\end{enumerate}











